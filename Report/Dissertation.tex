\documentclass[11pt,a4paper]{report}

%\renewcommand{\familydefault}{\sfdefault}

%\usepackage{phv} % Helvetica Font
\usepackage[utf8]{inputenc}
\usepackage{amsmath} 
\usepackage[left=3.00cm, right=2.54cm, top=2.54cm, bottom=2.54cm]{geometry}


 
\title{Parallel Recurrent Cerebellar Controllers For Controlling a Time-Varying Plant}
\author{Kiyan Maheri}
\date{2017} 
 
\begin{document}

\begin{titlepage}
\maketitle
\end{titlepage}

\section{Introduction}
It can be observed in every day life that animals perform very well in unstructured and uncetain environments. Animals achieve this feat whilst having highly nonlinear and time-varying components This attests to the quality and versatility of our biological motor-control systems. (Using biorobotic components and control) Applying the control techniques used in animals to robotics would be an improvement on current robot performance.\par


A challenge that the motor-control system faces is varying plant dynamics. This change could be slow-acting (e.g. growth, healing or slow damage) or fast-acting (picking up an object, sudden damage, muscle fatigue). (How much of this applies to biorobotics) The motor-control system has to operate with some degree of uncertainty, as often not enough sensory information is available to predict plant dynamics \textit{a priori}.\par 


Neuroscience studies show that the brain has two mechanisms to improve control: Motor Learning and Distributed Control. In control engineering terms these can be thought of as equivalent to Adaptive Control and Modular Control.(Expand on purpose of two mechanisms?)\par 

The Recurrent Cerebellar control model is a biologically plausible adaptive controller with a novel solution to the motor error problem (see Literature Review). As a feedforward controller it is impervious to the large sensory delays found in the motor-control system. This work explores the use of the Recurrent Cerebellar model in a modular control architecture. To this end a new modular controller was designed for controlling a (time-varying/context-switching/discretely switching) plant. \par


The project will also investigate mechanisms to increase the range of the controller by adding
and training new controller modules when faced with a novel context.\par

* We don't know exactly how we work, but Robotics experiments could illuminate mechanisms

%\section{Literature Review}
        %\subsection{Cerebellum}
        %\subsection{Multiple controllers}
%\section{Methods}
%\section{Results}
%\section{Discussion}
%\section{Conclusion}
\end{document}


